\chapter{Introduction} \label{chapter1}
Your first chapter is probably an introduction. But who knows. Check out Eq.\ (\ref{that_right_triangle_rule})!
Note that after Eq.\ and Fig.\ you want to use `.\textbackslash'  to use a single sized space. Otherwise,
latex will interpret it as the end of a sentence and put additional white space in between `Eq.' and 
`(\ref{that_right_triangle_rule})'.
Actually, you might want to input a tilde (\texttt{\~}) (a non-breaking space) instead of a space, so that the `Eq.' and number don't break across a line.
Or you could use commands from \texttt{hyperref} or \texttt{cleveref}.
See \cref{chapter2} for that...

By the way, I strongly recommend splitting up your paragraphs so that you have each sentence on its own line.
This makes the diffs much easier to read, which makes version control more useful.

\begin{equation}
a^2 + b^2 = c^2 \label{that_right_triangle_rule}
\end{equation}

The Physics Department recommends that the first chapter of the thesis be 
a succinct summary of the entire thesis, including in particular:

\begin{enumerate}
  \item a brief review of the field prior to the thesis research to provide context
  \item a presentation of the goals and motivations of the thesis research 
  \item a clear description of what the student has achieved in the thesis research
 (primarily written in the first person singular, but with due credit to
 others as appropriate). This description should refer back to (1) and clearly indicate the relation
 to prior work.
\end{enumerate}
It may also make sense to add suggestions for how to best build upon the thesis research in future work. Otherwise these suggestions should appear in the conclusion of the thesis.
